\documentclass[14pt]{article}

\begin{document}
\title{Some Notes on Graph Mining}
\maketitle

\section{Into2GraphMining}
\begin{enumerate}
 \item A graph is said to be connected if there is path between every pair of vertices
 \item Two graphs $G_1(V_1,E_1)$ and $G_2(V_2,E_2)$ are said to be isomorphic if they are topologically identicle, which means a mapping from $V_1$ to $V_2$ exists so that each edge $E_1$ is mapped to a single edge in $E_2$ and vice-versa.
 \item Frequent subgraph mining (FSM)
  \begin{itemize}
   \item Given a set of undirected and labeled graphs ($D$) and a support threshold $\sigma$, find all connected and undirected graphs that are subgraphs in at least $\sigma \times D$ of input graphs.
  \end{itemize}
\end{enumerate}

\section{Complex networks tools for analyzing networks (R+igraph)}
\begin{enumerate}
 \item \texttt{igraph} can be used to handle undirected and directed graphs. It includes implementations for classic graph theory problems like minimum spanning trees and network flow and community structure search.
 \item Procefures for analyzing network
  \begin{itemize}
   \item Create a graph object
   \item Layout the network: use \texttt{igraph: tkplot}
   \item Ranking: use  \texttt{igraph: page.rank}
   \item Metrics
    \begin{itemize}
     \item \texttt{igraph: diameter(g)}
     \item \texttt{igraph: graph.density(g)}, i.e., $\frac{No.edges}{No.vertex \times (No.vertex-1)}$
     \item \texttt{igraph: average.path.length(g)}
     \item \texttt{igraph: transitivity(g)}
    \end{itemize}
   \item Community detection
   \item Export
  \end{itemize}

\end{enumerate}


\end{document}